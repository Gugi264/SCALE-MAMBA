\mainsection{System Commands}
This sections assume you kind of understand the SCALE system
and have already programmed a bit in MAMBA. Eventually we will
make it all self-contained. But for now lets assume you are
an expert...

\subsection{System Control Commands}
\func{start_clock(x)}
Initializes the timer number $x$.

\func{stop_clock(x)}
Stops the timer number $x$.
\begin{lstlisting}
   start_clock(3);
   ....
   stop_clock(3);
\end{lstlisting}

\func{require_bit_length(x)}
For some reason you program might require a minimum prime
length. This command stores the maximum prime length which you
signal, and then emits a single \verb|REQBL| command at the
beginning of the output assembler.
Thus the Rust code
\begin{lstlisting}
    ...
    require_bit_length(10);
    ...
    require_bit_length(20);
    ...
\end{lstlisting}
creates assembler with a single \verb|REQBL 20| as the first 
instruction.

\func{crash()}
Crashes the system.

\func{restart()}
Executes the restart machinery.

\func{clear_memory()}
Executes a clear memory command.

\func{clear_registers()}
Executes a clear registers command.

\func{open_channel(x)}
Opens the communication channel to the outside world, returning
the appropriate value as an $\ints$.
\begin{lstlisting}
    let ans = open_channel(10_i32);
\end{lstlisting}

\func{close_channel(x)}
Closes the communication channel to the outside world.
\begin{lstlisting}
    close_channel(10_i32);
\end{lstlisting}


\subsection{Stack Operations}
Recall (from the main Scale documentation) there is a stack for all 
the five basic types $\ints$, $\SecretI$, $\ClearModp$, $\SecretModp$ and $\SecretBit$.
As the main Rust compiler also uses these stacks for function
calls you need to be very careful when using the stacks.
Thus you need to enclose them in an \verb|unsafe| block.
The basic methodology to use the stacks is the same for all five
types, so we just give one example here.

\begin{lstlisting}
    let one = SecretI64::from(1);
    let two = SecretI64::from(2);
    let three = SecretI64::from(3);
    let four = SecretI64::from(4);

    unsafe {
        SecretI64::push(one);
        SecretI64::push(two);
        let sp = SecretI64::get_stack_pointer();
        let c = SecretI64::peek(sp - 1);       // Peek relative to pos 0
        let cc = SecretI64::peek_from_top(1);  // Peek relative to pos stack_ptr
        SecretI64::poke(sp, &three);           // Poke relative to pos 0
        SecretI64::poke_from_top(1, &four);    // Poke relative to pos stack_ptr
        let d = SecretI64::pop();
        let e = SecretI64::pop();
    }
\end{lstlisting}

\subsection{Printing}
Access to the printing commands is performed via the
\verb|print!| command. 
Currently, only strings and $\ClearModp$ types can be output in this way.
\begin{lstlisting}
    let ca = ClearModp::from(1);
    let cb = ClearModp::from(2);
    print!("hello ", ca, " world ", cb, "\n");
\end{lstlisting}
If you cannot be bothered to add the newline at the end 
you can also use \verb|println!|
\begin{lstlisting}
    let ca = ClearModp::from(1);
    let cb = ClearModp::from(2);
    println!("hello ", ca, " world ", cb);
\end{lstlisting}


\subsection{Circuits and Local Functions}
For details of how these work in general see the main SCALE manual.
If you want to add your own circuits/local functions see the
files
\begin{itemize}
\item \verb|WebAssembly/scale_std/src/circuits.rs|
\item \verb|WebAssembly/scale_std/src/local_functions.rs|
\end{itemize}
which should be relatively self-explanatory as to how to add
new functions to the underlying system.

\subsubsection{Inbuilt Circuits}
The inbuilt circuits shipped with SCALE can be called using the
following function signatures (the ones associated to floating point
operations can be accessed via the \verb|SecretIEEE| type).
\begin{lstlisting}
    pub fn AES128( key128: [SecretI64; 2], mess: [SecretI64; 2]) 
                  -> [SecretI64; 2]

    pub fn AES192( key128: [SecretI64; 3], mess: [SecretI64; 2]) 
                  -> [SecretI64; 2]

    pub fn AES256( key128: [SecretI64; 4], mess: [SecretI64; 2]) 
                  -> [SecretI64; 2]

    pub fn SHA3( istate: Array<SecretI64, 25>) -> Array<SecretI64, 25>

    pub fn SHA256( mess: Array<SecretI64, 8>, state: Array<SecretI64, 4>) 
                  -> Array<SecretI64, 4>

    pub fn SHA512( mess: Array<SecretI64, 16>, state: Array<SecretI64, 8>) 
                  -> Array<SecretI64, 8>

    pub fn IEEE_add( input: [SecretI64; 2] ) -> SecretI64

    pub fn IEEE_mul( input: [SecretI64; 2] ) -> SecretI64

    pub fn IEEE_div( input: [SecretI64; 2] ) -> SecretI64

    pub fn IEEE_eq( input: [SecretI64; 2] ) -> SecretI64

    pub fn IEEE_f2i( input: SecretI64 ) -> SecretI64

    pub fn IEEE_i2f( input: SecretI64 ) -> SecretI64

    pub fn IEEE_sqrt( input: SecretI64 ) -> SecretI64

    pub fn IEEE_lt( input: [SecretI64; 2] ) -> SecretI64
\end{lstlisting}
For example, to call the AES-128 circuit you would execute:
\begin{lstlisting}
    #[inline(always)]
    fn main() {

        let zero = SecretI64::from(0);
        let one = SecretI64::from(1);
        let mone = SecretI64::from(-1);
    
        let key128: [SecretI64; 2] = [zero, mone];
        let mess: [SecretI64; 2] = [mone, one];
    
        let ciph = AES128(key128,mess);
    }
\end{lstlisting}



\subsubsection{Inbuilt Local Functions}
The calling of Local Functions works in much the same way.
If you look at the implementation of matrix multiplication of
two matrices\footnote{See later for the matrix types} 
of type $\ClearModp$ in \verb|WebAssembly/scale_std/src/local_functions.rs|
you will find the implementation:
\begin{lstlisting}
    const C_MULT_C: u32 = 0;

    #[inline(always)]
    #[allow(non_snake_case)]
    pub fn Matrix_Mul_CC<const N: u64, const L: u64, const M: u64>
                        ( A: &Matrix::<ClearModp, N, L>,
                          B: &Matrix::<ClearModp, L, M>
                        ) -> Matrix::<ClearModp, N, M> {
        let (col,row,C)= unsafe { execute_local_function!(C_MULT_C(
            N as i64,
            L as i64,
            *A,
            L as i64,
            M as i64,
            *B
        ) ->
            i64,
            i64,
            Matrix::<ClearModp, N, M>
        ) };
        if row != (N as i64) || col != ( M as i64) {
            crash();
        }
        C
    }
\end{lstlisting}
We explain this now, as you can replicate this to add your own
Local Functions:
\begin{itemize}
\item The `unsafe' part calls the Local Function with index zero.
\item The arguments are pushed onto the stack in the order
$N$, $L$, $A$, $L$, $M$ and $B$.
\item The operation is then called, with the results popped
off the stack in the order $\mathsf{col}, \mathsf{row}$
and $C$.
\item After some doubly checking the matrix $C$ is returned to the 
caller.
\end{itemize}
The caller code is given by
\begin{lstlisting}
    let mut A = Matrix::<ClearModp, N, L>::uninitialized();
    let mut B = Matrix::<ClearModp, L, M>::uninitialized();

    ... Set data in A and B ...

    let C = Matrix_Mul_CC(&A, &B);
\end{lstlisting}
Currently there are four Local Functions defined, with
the signatures  (the ones in the main SCALE manual associated to floating point
operations can be accessed via the \verb|ClearIEEE| type directly).
\begin{lstlisting}
    pub fn Matrix_Mul_CC<const N: u64, const L: u64, const M: u64>
                    ( A: &Matrix::<ClearModp, N, L>,
                      B: &Matrix::<ClearModp, L, M>
                    ) -> Matrix::<ClearModp, N, M> 

    pub fn Matrix_Mul_SC<const N: u64, const L: u64, const M: u64>
                    ( A: &Matrix::<SecretModp, N, L>,
                      B: &Matrix::<ClearModp, L, M>
                    ) -> Matrix::<SecretModp, N, M>

    pub fn Matrix_Mul_CS<const N: u64, const L: u64, const M: u64>
                    ( A: &Matrix::<ClearModp, N, L>,
                      B: &Matrix::<SecretModp, L, M>
                    ) -> Matrix::<SecretModp, N, M>

    pub fn Gauss_Elim<const N: u64, const M: u64>
                 ( A: &Matrix::<ClearModp, N, M>,
                 ) -> Matrix::<ClearModp, N, M>
\end{lstlisting}

