\mainsection{Changes}

This is v1.3 of the Rust-based implementation system for SCALE.

\subsection{Current Known Bugs}
\begin{enumerate}
\item Operator arithmetic operations for  \verb|ClearInteger<K>| and \verb|SecretInteger<K>|.
\item Functions with multiple return values can cause a problem for the compiler, especially
when they are not-inlined. Thus avoid these. If you get weird behaviour this might
be the reason.
\item When compiling some loops the Rust-to-Wasm compiler sometimes creates
	registers which are `assumed' to be zero-assigned on first use.
	No idea why this happens, but you can see it by compiling SCALE with
	the flag \verb|-DDEBUG|. This seems to cause no effect for normal
	programs, but will cause problems (possibly) for programs called
	via the \verb|restart| mechanism. To avoid this problem we issue
	a \verb|clear_register| command at the start of each program,
	and a Rust \verb|restart| command deletes the memory allocated
	for testing and the wasm stack before executing the SCALE
	\verb|restart| opcode.
\end{enumerate}

\subsection{Changes in version 1.3 from 1.2}
\begin{enumerate}
\item A little more of the emulation, in the testing mode, is correct.
      This aids a bit with debugging.
\item A bug related to function calling has been fixed.
\item Optimized \verb|SecretBit| operations.
\item $\sqrt{x}$ and $\log(x)$ operations have been implemented for \verb|ClearFloat|
	and \verb|SecretFloat| data types. Note, this is a definite
	improvement on Mamba, where they were not implemented at all.
\item A simple version of ORAM style operation is now implemented. This
	enables reading/writing to \verb|Array|s and \verb|Slice|s
	given a $\SecretModp$ index.
\item We can now support \verb|Array|s and \verb|Slice|s of 
	\verb|ClearFloat| and \verb|SecretFloat| values.
	By extension by copying how the allocators are done for
	these types, one can also support arrays and slices of
	arbitrary types which consist of multiple elements 
	of the same base-type.
	Eventually this needs to be extended to types which consist
	of elements of different base types.
\item You no longer need to define channels and players using the
      keywords \verb|Channel| and \verb|Player|. 
      This means you channels and players are normal \verb|i64| values,
      and hence can be programmatically defined, i.e. via loop values etc.
\end{enumerate}

\subsection{Changes in version 1.2 from 1.1}
\begin{enumerate}
\item Decision has been made to not port the vectorized instructions over
	to Rust. The reason is that they require changes to the assembler,
	the compiler {\em and} require the programmer to jump through hoops
	to use them. Instead we are implementing optimization pipelines
	and changes to the runtime to allow the same `effects', at
	time critical components, to be applied to programs without the 
	need for the user to explicitly work this out and program this
	themselves.
	The first change is a merging of \verb|TRIPLE| and \verb|SQUARE|
	instructions, which is oblivious to the user, but gives for
	some programs a ten percent performance improvement.
\item  The second change is a `vectorized' private input and output
	routine for Arrays and Slices, which under-the-hood works
	via the SCALE memory (as opposed  to the usual SCALE method
	of vectorizations by having contiguous register names).
\item The third change is some more advanced operations on Arrays and Slices,
	which have been propogated into our Rust standard library.
\item The overall effect is that programs in the Rust pipeline
	now are about as efficient as the equivalent programs in the
       Mamba pipeline.
\item Added data types \verb|ClearFloat<V,P>| and \verb|SecretFloat<V,P>|
	to replicate the $\mathsf{cfloat}$ and $\mathsf{sfloat}$ in Mamba.
	The maths library for these datatypes is not yet fully implemented.
\item Almost all data types now have a means for generating random values
	within them.
\item Some operations on integers and bit operations have had their
	API changed a little.
\item Modified the way \verb|restart()| is compiled to avoid a bug when using the
	restart functionality and rust programs, for a similar reason we issue a
      \verb|clear_register| operation now at the start of each program.
	See the `Known Bugs' section above. \nps{There is still a bug here.
		Will fix soon}
\end{enumerate}

\subsection{Changes in version 1.1 from 1.0}
\begin{enumerate}
\item Some functions which should have been {\bf unsafe} are now marked as {\bf unsafe}.
\item Added documentation of some functions related to bit processing.
\item Added in comparison member functions for the \verb|ClearInteger<K>| type.
\item Fixed some arithmetic bugs in \verb|ClearInteger<K>| and \verb|SecretInteger<K>| types.
\item Round complexity between Rust pipeline and the old Mamba pipeline is the same now for some test benchmarking functions.
\item The \verb|ClearIEEE| class can now process arithmetic operations
	as well as having a full math library, which is relatively fast.
	Thus the \verb|ClearIEEE| type should be preferred for all
	{\em clear} operations on floating point values.
\item A full math library for \verb|SecretIEEE| is implemented.
\item The crate for \verb|ClearIEEE| and \verb|SecretIEEE| is now
	called simply \verb|scale_std::ieee|
\item Added fixed point operations with new data types
\verb|ClearFixed<K,F>| and \verb|SecretFixed<K,F>|.
Note, the \verb|SecretFixed<K,F>| is generally much faster than 
the \verb|SecretIEEE| type.
\item Conversion between \verb|ClearFixed<K,F>| and \verb|ClearIEEE|
	is possible by means of the respective \verb|from| functions.
\item Arrays/Slices can now be created for the datatypes
	\verb|ClearIEEE|,  \verb|SecretIEEE|, \verb|ClearFixed<K,F>| and \verb|SecretFixed<K,F>|.
\item Memory management is now dynamic and done via the SCALE runtime, thus the need to define memory size at the start of a SCALE Rust program has gone.
\item \verb|Array| and \verb|Slice| are now more robust and closer to what the standard
        Rust \verb|Vec| type does.
	Note use of \verb|get| on an \verb|Array| and \verb|Slice| costs at runtime
	so use  \verb|get_unchecked| where possible.
\end{enumerate}
