\subsection{SecretModp}
This is the same as the $\sint$ in the underlying SCALE virtual machine.

\subsubsection{Conversion of Data}
\func{SecretModp::from(a)}
This takes an $\sint$ integer value $a$ and loads it into a value
of type $\SecretModp$.
You can also pass in an argument of a type $\ints$, $\ClearModp$,
$\SecretBit$ and $\SecretI$.
\begin{lstlisting}
    let sa = SecretModp::from(ConstI32::<8>);

    let aa: i64 = 10;
    let mut a2=SecretModp::from(aa);

    let cint_a = ClearModp::from(ConstI32::<10>);
    let sint_a = SecretModp::from(cint_a);

    let sb = SecretBit::from(false);
    let ssb = SecretModp::from(sb);

    // See the main SCALE manual for the semantics of this conversion
    let si = SecretI64::from(ConstI32::<10>);
    let ssi = SecretModp::from(si);
\end{lstlisting}

\func{SecretModp::from_unsigned(a)}
The above conversion from $\SecretI$ to $\SecretModp$ is a signed conversion.
To get an unsigned conversion you perform.
\begin{lstlisting}
    let si = SecretI64::from(ConstI32::<-10>);
    let ssi = SecretModp::from_unsigned(si);
\end{lstlisting}

\subsubsection{IO}
The IO class functions for $\SecretModp$ datatypes can be access via the following commands:

\func{SecretModp::private_input(p,c)}
Loads an $\SecretModp$ from the channel $c$ and player $p$.
\begin{lstlisting}
    let a=SecretModp::private_input(3, 10);
\end{lstlisting}

\func{a.private_output(p,c)}
Writes an $\SecretModp$ to channel $c$ and player $p$.
\begin{lstlisting}
   let v = SecretModp::from(ConstI32::<1>);
   v.private_output(0, 10);
\end{lstlisting}

\subsubsection{Other Operations}

\func{SecretModp::randomize()}
Produces a (secret) random number in the range $[0,\ldots,p-1]$,
where $p$ is the prime underlying the $\SecretModp$ type.
This random number is the `same' for all players.
\begin{lstlisting}
   let a = SecretModp::randomize();
\end{lstlisting}

\func{SecretModp::get_random_bit()}
Produces a (secret) random number in $\{0,1\}$.
This random bit is the `same' for all players.
\begin{lstlisting}
   let a = SecretModp::get_random_bit();
\end{lstlisting}

\func{SecretModp::get_random_square()}
Produces a (secret) random number $a$ in the range $[0,\ldots,p-1]$,
where $p$ is the prime underlying the $\SecretModp$ type,
as well as its square $b$.
\begin{lstlisting}
   let (a,b) = SecretModp::get_random_square();
\end{lstlisting}

\func{SecretModp::get_random_triple()}
Produces two (secret) random numbers $a$ and $b$ in the range $[0,\ldots,p-1]$,
where $p$ is the prime underlying the $\SecretModp$ type,
as well as their product $c$.
\begin{lstlisting}
   let (a,b,c) = SecretModp::get_random_triple();
\end{lstlisting}

\subsubsection{Memory Access}
\func{a.store_in_mem(x)}
To store data into memory location $50$ say you need to execute,
recalling that memory is accessed by $32$ bit values.
\begin{lstlisting}
    unsafe{ sa.store_in_mem(ConstU32::<50>); }
\end{lstlisting}
To store to a variable location you use
\begin{lstlisting}
    let x: i64 = 50;
    unsafe{ sa.store_in_mem(x); }
\end{lstlisting}

\func{SecretModp::load_from_mem(x)}
To load data from memory you do
\begin{lstlisting}
    let sa=SecretModp::load_from_mem(ConstU32::<50>);
    let x: i64 = 50;
    let sb=SecretModp::load_from_mem(x);
\end{lstlisting}


\subsubsection{Testing Data}
\func{x.test()}
In the SCALE target this takes the $\SecretModp$ value $x$,
applies a reveal operation to it, and then writes the
resulting $\ClearModp$ value into memory.

In the Test target this takes a value stored in the $\ClearModp$
memory saved on the last SCALE invocation, and compares it to
$x$. If the two values are the same it prints the value, and the
line number in the rust file where this was executed.
Otherwise it aborts.

\func{x.test_value(y)}
Test whether the value held in $x$ is the same as the value held in $y$.
This is the same as \verb|x.test()| except the value is compared to
$y$ and not the value emulated in the test environment.
