\subsection{Operators}

\subsubsection{Unary Operators}
The following unary operations are allowed on $\SecretBit$, $\SecretI$.
\begin{verbatim}
     !
\end{verbatim}
This operation inverts all bits (or just the one bit in the $\SecretBit$ case)
of its argument. There is no $!$ operation like in C on integers. If you want to
check whether all bits are zero in Rust, you need to use an equality operation.


\subsubsection{Binary Operators}
The following operations are allowed between $\SecretI$ and $\ints$
types:
\begin{verbatim}
     +   -   *   /
     ^   &   |
\end{verbatim}
The output is of type $\SecretI$ if any of the two operands are
$\SecretI$s, otherwise the output type is a $\ints$.

~\\

\noindent
The following operations are allowed between $\SecretModp$ and $\ClearModp$
types:
\begin{verbatim}
     +   -   *
\end{verbatim}
The output is of type $\SecretModp$ if any of the two operands are
$\SecretModp$s, otherwise the output type is a $\ClearModp$.

~\\
\noindent
The following operations are allowed between $\SecretModp$ and $\ClearModp$
types:
\begin{verbatim}
     /
\end{verbatim}
The output is of type $\SecretModp$ if the first operand is
$\SecretModp$s, otherwise the output type is a $\ClearModp$.
     {\bf Note:} The second operand must be a $\ClearModp$ or a \verb|ConstI32| at present.

~ \\

\noindent
The following operations are allowed between $\ClearModp$ and $\ints$ types:
\begin{verbatim}
    %   <<   >>  ^  &  |
\end{verbatim}
The operations are performed by lifting the $\ClearModp$ values to
the integers (in the range $[0,\ldots,p)$) and then performing
the operation over the integers.  The output is of type $\ClearModp$

~ \\

\noindent
The following operations are allowed between $\SecretBit$ types:
\begin{verbatim}
     ^   &   |
\end{verbatim}
The output is of type $\SecretBit$

~\\

\noindent
The following operations are allowed between $\ints$ types:
\begin{verbatim}
     ==    !=
     <   >   <=  >=
\end{verbatim}
The output is of type $\ints$/$\bool$ and therefore allows branching
on the resulting condition etc.


~ \\

\noindent
Between an $\SecretI$, $\SecretModp$ or $\ClearModp$ and an immediate operand (an \verb|ConstU32|)
you can execute shift-left and shift-right operations. These operators also exist if both operands
are $\ints$.
\begin{verbatim}
     <<    >>
\end{verbatim}
As in
\begin{lstlisting}
   let sb = sa << ConstU32::<3>;
\end{lstlisting}
For integers, the small example needs to use $\mathsf{black\_box}$ 
as otherwise the optimizer will optimize the shift and just store 
the end result.
\begin{lstlisting}
   let a = black_box(2);
   let b = black_box(1);
   let c = a << b
\end{lstlisting}

